\subsection{Node Split}
%% \begin{algorithm}
% 	\KwIn{Dependency Graph $G_d$}
% 	\KwOut{Simple Directed Graph $G_d$}
% 	\SetKwFunction{FindPair}{FindPair}
% 	\SetKwFunction{UpdateGraph}{UpdateGraph}
% 	\SetKwFunction{RecoverTimeLogic}{RecoverTimeLogic}
% 	\While{is not Empty}{	
% 	$Queue$ $\leftarrow$ \FindPair{$G_d$}
% 	\While{Queue is not Empty}{
% 			$P$ $\leftarrow$ $Queue.poll()$ \\
% 			$s \leftarrow P.source$\\
% 			$t \leftarrow P.target$\\
% 			\If{Set contains s $or$ Set contains t}{continue} \\
% 			$list1 \leftarrow$ list of outgoing edges of $s$ whose target is not $t$ \\
% 			$list2 \leftarrow$ V.list\\
% 			\UpdateGraph($G_d$,$list1$,$list2$)\\
% 			add $S$ to $Set$\;
% 		}	
% 		\caption{Node Split}
% 	\label{alg:split}
% \end{algorithm}

% \begin{algorithm}
%     \KwIn{Dependency Graph(G)}
%     \KwOut{A simple directed graph}
%           $Queue \leftarrow FindPair(G)$ \\
%           \While{Queue is not Empty}{
%             \While{Queue is not Empty}{
%                 NodePair \leftarrow $Queue.poll()$ \\
%                 \If{NodePair.u or NodePair.v has been splited}{
%                     continue
%                 }
%                 UpdateGraph(NodePair.u) \\
%             }
%             Queue \leftarrow FindPair(G) \\
%           }
%           CheckCasualRelation(G)\\
%         \caption{Node Split}
%     \label{alg:split}
% \end{algorithm}

% In some situations, there will be some edges can not be merged together because of different system event types. Assume nodes $u$ and $v$ are a pair of nodes in the graph with multiple edges between them. We borrow the idea of static single assignment form (SSA)~\cite{nielson2004principles} that splits a node $u$ into multiple nodes, where each node has only one outgoing edge pointing to node $v$. The nodes generated by split step will inherit the same incoming edges of the original \textit{u}. After this step, we could use a transition matrix to present the weights of edges. Algorithm~\ref{alg:split} shows the process of \emph{Node Split}, we also need several help functions to split nodes:
% \begin{itemize}
%     \item \emph{FindPair}: This function is used to find the pair of nodes who are still connected by multiple edges whose system event are different after \emph{Edge Merge}. The return value a is queue of the node pair need to be splited.
%     \item \emph{UndateGraph}: This function is used to split the node(\eg $u$).
%     \item \emph{CheckCasualRelation}: Because the the copy of $u$ inherits all the incoming edges, some incoming edges may break the causal relationship(\eg the start time of incoming edges is bigger than the start time of outgoing edges). This function will remove the incoming edges that break the causal relationship. 
% \end{itemize}
