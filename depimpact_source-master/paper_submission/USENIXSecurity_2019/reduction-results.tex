


% \begin{table}[]
\centering
        \caption{Filtering Results}
        \label{tab:filter}
        \resizebox{0.45\textwidth}{!}{%
            \begin{tabular}{|l|r|r|r|}
            \hline
            \thead{Cases} & \thead{\#Critical Edges} & \thead{Missing Point} & \thead{\#Non-critical Edges at Missing Point}\\\hline
            2File                     & 3                        & 9.49          & 0                                     \\\hline
            3File                     & 4                        & 6.49          & 0                                     \\\hline
            Python-wget               & 4                        & 5.19          & 1                                     \\\hline
            Python-wget-unzip         & 8                        & $<0.01$          & 60                                    \\\hline
            Shell-script              & 4                        & 3.15          & 1                                     \\\hline
            Shell-wget                & 4                        & 0.04          & 6                                     \\\hline
            Shell-wget-unzip          & 6                        & 2.83          & 3                                     \\\hline
            USB-merge                 & 6                        & 2.11          & 0                                     \\\hline
            curl                      & 4                        & 12.80         & 1                                     \\\hline
            scp                       & 3                        & 8.29          & 4                                     \\\hline
            wget                      & 2                        & 6.20          & 29                                    \\\hline
            command-injection-c1 & 2                        & 17.00         & 49                                    \\\hline
            command-injection-c2 & 3                        & 286.50        & 0                                     \\\hline
            data-leakage             & 5                        & 302.89        & 0                                     \\\hline
            password-crack-c1    & 2                        & 9.00          & 34                                    \\\hline
            password-crack-c2    & 4                        & 14.20         & 1                                     \\\hline
            password-crack-c3    & 4                        & $<0.01$          & 57                                    \\\hline
            penetration-c1         & 3                        & 0.02          & 5                                     \\\hline
            penetration-c2         & 11                       & $<0.01$          & 21                                    \\\hline
            vpnfilter-c1          & 2                        & 4.67          & 12                                    \\\hline
            vpnfilter-c2          & 3                        & 3.60          & 15                                   \\\hline 
            \thead{average}    & 4                        & 32.92          &14.24 \\\hline
            \end{tabular}
        }
\end{table}
\begin{figure}[!ht]
    \centering
    \includegraphics[width=0.48\textwidth]{figs/fig:edge-thresh.pdf}
    \caption{Effectiveness of filtering. The percentage of edges remaining after filtering drops significantly at $T = 0.10$ and remains stable below 10\%.}
    %\dcaption{The percentage of edges remaining after filtering drops significantly at $T = 0.10$ and remains stable below 10\%.}
    \label{fig:edge-thresh}
\end{figure}


\begin{figure}[!ht]
    \centering
    \includegraphics[width=0.48\textwidth]{figs/fig:cdf.pdf}
    \caption{Critical edge loss from filtering. All missing points are distributed between $T = 0.25$ and $T = 1.0$.}
    %\dcaption{All missing points distribute between $T = 0.25$ and $T = 1.0$.}
    \label{fig:cdf}
\end{figure}


\subsubsection{Attack Sequence Reconstruction}
\label{subsec:graphreduction}


The important goal of attack sequence reconstruction is to filter out as many irrelevant edges as possible while preserving critical edges in the dependency graph.
Based on the edge weights (\cref{subsec:weight-computation}), \tool hides non-critical edges whose weights are below a threshold.
To provide a guidance on choosing this threshold, we test the filtering performance on all the attacks studied in \cref{subsec:reputation-results}
by using a range of values from 0.05 to 0.95 with a pace of 0.05 as the thresholds. 
% The edges whose weights are below this number are hidden.

\cref{fig:edge-thresh} shows the average percentage of remaining edges after edge filtering. We observe that when the threshold reaches 0.10, the average percentage of remaining edges is 9.8\%, and further increasing the threshold from 0.10 to 0.90 only results in a slightly increased amount of pruned edges (2.2\%). Such results indicate that most of the edges having reputation scores below 0.10 or above 0.90. 

While a higher threshold can hide more irrelevant edges, it may cause the loss of critical edges as well.
Thus, we define the \emph{missing point} as the greatest threshold that preserves all the critical edges for a dependency graph, 
and measure the missing points for all the attacks.
\cref{fig:cdf} shows the cumulative distribution of the missing points for all the attacks.
We observe that:
(1) all missing points are greater than 0.25,
and (2) about 80\% of the missing points are greater than 0.90.

Combining the results in \cref{fig:edge-thresh} and \cref{fig:cdf},
we can conclude that the reputation scores of almost all the critical edges are above 0.90, while the reputation scores of most of the non-critical edges are below 0.10.
Such results clearly demonstrate the effectiveness of \tool in leveraging the discriminative weights (\cref{subsec:weight-computation}) to distinguish critical edges from non-critical edges.
Furthermore, based on \cref{fig:cdf}, we can suggest an optimal range of the threshold: $\lbrack 0.1,0.25 \rbrack$.
  
  
  
\eat{
% (1) Two cases (\emph{command-injection-c2}, \emph{data-leakage}) have extremely high missing points ($T_w > 200$);
% (2) 5 out of 21 cases lost critical edges at $T_w = 2$. However, in these 5 cases, 2 of them(\emph{Shell-wget},\emph{penetration-c1}) already have less than 10 non-critical edges at missing point and 3 of them also have significant reduction in edge numbers(\cref{tab:filter}).
% (3) A plateau exists before $T_w = 2$ at a rate of 24\%. This indicate most of the cases have a missing point greater than $T_w = 2$, which proves the efficacy of our weights to differ critical edges from non-critical edges.
}

\eat{
\begin{table}[]\
\centering
    	\caption{Graph reduction results for PHASEs I and II}
    	\label{tab:Reduction}
    	\resizebox{0.48\textwidth}{!}{%
            \begin{tabular}{|l|r|r|r|r|}
            \hline
            \thead{Attacks}                & \thead{PH.I V} & \thead{PH.I E} & \thead{PH.II V} & \thead{PH.II E} \\ \hline
            3File      & 225                  & 25211             & 225            & 596         \\ \hline
            2File             & 221                  & 21100             & 221            & 537         \\ \hline
            curl                  & 212                  & 15142             & 212            & 460         \\ \hline
            shell\_script           & 229                  & 21124             & 229            & 637         \\ \hline
            python\_wget          & 162                  & 10597             & 162            & 283         \\ \hline
            scp                   & 60                   & 7912              & 60             & 78          \\ \hline
            shell\_wget           & 115                  & 4998              & 115            & 158         \\ \hline
            wget                  & 217                  & 19078             & 217            & 500         \\ \hline
            command-injection-c1      & 51                   & 65                & 51             & 51          \\ \hline
            command-injection-c2      & 1438                 & 9713              & 1438           & 2738        \\ \hline
            data-leakage       & 4252                 & 151200            & 4252           & 4261        \\ \hline
            password-crack-c1 & 35                   & 731               & 35             & 36          \\ \hline
            password-crack-c2 & 61                   & 2477              & 61             & 73          \\ \hline
            password-crack-c3 & 25                   & 58426             & 25             & 36          \\ \hline
            penetration-c1 & 60                   & 314               & 60             & 63          \\ \hline
            penetration-c2 & 30                   & 139               & 30             & 68          \\ \hline
            vpnfilter-c1      & 14                   & 274               & 14             & 14          \\ \hline
            vpnfilter-c2      & 16                   & 598               & 16             & 18          \\ \hline
            avg & 412.39 & 19394.39 & 412.39 & 589.28 \\ \hline
            \end{tabular}
        }
    % \dcaption{ .}
\end{table}

Thus, we evaluate the attack sequence reconstruction via two aspects:
(1) graph reduction and (2) revealing critical edge.

\myparatight{Graph Reduction}
\cref{tab:Reduction} shows the reduction in terms of the number of nodes (Columns ``PH.I V'' and ``PH.II V'') and in the number of edges  (Columns ``PH.I E'' and ``PH.II E'') after causality analysis in Phase I (\cref{subsec:graph-generation}) and edge merge in Phase II (\cref{subsec:graph-preprocessing}).
As we can see, the reduction is significant: the biggest graph generated by \tool only contains 4252 nodes and 4261 edges (``data-leakage'' in \cref{tab:Reduction}) considering the 24-hour log contains about \emph{2 billion} events.

\myparatight{Revealing Critical Edge}
}