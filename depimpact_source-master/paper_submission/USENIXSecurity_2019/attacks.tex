\subsubsection{APT Attacks}
\label{subsubsec:attack-cases}

We performed five APT attacks that capture the important traits of APT attacks depicted from the Cyber Kill Chain framework~\cite{cyberkillchain}. 
Note that an APT attack consists of a series of steps, and some steps may not be captured by system monitoring (\eg user inputs and inter-process communications).
Such limitations can be addressed by employing more powerful auditing tools, but it is out of the scope of this paper.
Thus, we identified 10 key steps that are related to POI entities for our evaluations in the five APT attacks.


\myparatight{APT Attack 1: Zero-Day Penetration to Target Host}
The scenario emulates the attacker's behavior who penetrates the victim's host
leveraging previously unknown Zero-day attack. Zero-day vulnerabilities are
attack vectors that previously unknown to the community, therefore allow the
attacker to put their first step into their targets. In our case, we assume that
the {\tt bash} binary in victim's host is outdated and vulnerable to shellshock~\cite{shellshock}. The victim computer hosts web service that has
CGI written as BASH script. The attacker can run an arbitrary command when she
passes the specially crafted attack string as one of environment variable. Leveraging the vulnerability, the attacker runs a series of remote commands to
plant and run initial attack by: (1) transferring the payload (\emph{penetration-c1}), (2) changing its permission, and (3) running the payload to bootstrap its campaign (\emph{penetration-c2}).
% As a lateral movement, the
% attacker downloads (4) a password cracker program from outside run it against
% the shadow password files. 


\myparatight{APT Attack 2: Password Cracking After Shellshock Penetration}
% Once breaks into the system, the attacker can launch different malicious
% behaviors (\eg password cracking, information leakage, denial of services). 
After initial shellshock penetration, the attacker first connects to Cloud services (\eg
Dropbox, Twitter) and downloads an image where C2 (Command and Control) host's IP address is encoded in EXIF metadata (\emph{password-crack-c1}). The behavior is a common practice shared by APT attacks~\cite{hammertoss,vpnfilter} to evade the network-based detection system based on DNS blacklisting.

Using the IP, The malware connects to C2 host. C2 host directs the malware to take
some lateral movements, including a series of stealthy reconnaissance maneuvers. 
In this stage, the attacker generally takes a number of actions. Among those, we emulate the password cracking attack. The attacker downloads password cracker payload (\emph{password-crack-c2}) and runs it against password shadow files (\emph{password-crack-c3}).

\myparatight{APT Attack 3: Data Leakage After Shellshock Penetration}
After lateral movement stage, the attacker attempts to steal all the valuable assets from the host. 
This stage mainly involves the behaviors of local and remote file system scanning activity, copying and compressing of important files, and transferring to its C2 host.
The attacker scans the file system, scrap files into a single compress file and transfer it back to C2 host (\emph{data-leakage}).

\myparatight{APT Attack 4: Command-line Injection with Input Sanitization Failures}
Different from the previous shellshock case, a program may contain vulnerabilities introduced by developer errors and this can also be a initial attack vector that invites the attacker into their target systems. To represent
such cases, we wrote an web application prototype that fails to sanitize inputs for a certain web request, hence allows Command line Injection attack. 
Our prototype service mimics the Jeep-Cherokee attack case~\cite{miller:remote:2015} which implements a remote access using the conventional web service API that
internally uses DBUS service to run the designated commands. 
Due to the developer mistake, the web service fails to sanitize the remote inputs, the attacker can append arbitrary commands followed by semi-colon({\tt;}). 
Leveraging this vulnerability, we can download backdoor program (\emph{commend-injection-c1}) and collect sensitive data (\emph{command-injection-c2}).

\myparatight{APT Attack 5: VPNFilter}
We prototyped a famous IoT attack campaign; VPNFilter malware~\cite{vpnfilterschenier}, which infected millions of dozens of different IoT devices exploiting a number of
known or zero-day vulnerabilities~\cite{vpnfilter1,vpnfilter2}. The attack's
significance lies in how the malware operates during its lateral movement stage following its initial penetration. 
The campaign employs up-to-date hacker practices to bypass conventional security solutions based on static blacklisting approaches and has an architecture to download plug-in payload on-demand, at run-time. 
We prototyped the malware referring to one of its sample for x86 architecture~\cite{vpnfilterx86}. 

The VPNFilter stage 1 malware accesses a public image repository to get an image. In the EXIF metadata of the image, it contains the IP address for the stage 2 host (\emph{vpnfilter-c1}). It downloads the VPNFilter stage2 from the stage2 server, and runs it (\emph{vpnfilter-c2}).


