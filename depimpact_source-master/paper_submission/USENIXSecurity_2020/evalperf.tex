%\begin{table}[!t]
\centering
\caption{Performance statistics of \tool}
\label{tab:runtime}
\resizebox{0.48\textwidth}{!}{%
\begin{tabular}{|l|r|r|r|r|r|}
\hline
            \textbf{Case}         & \textbf{Causality} & \textbf{Edge Merge} & \textbf{Node Split} & \textbf{Weight} & \textbf{Rep. Propagation} \\ \hline
penetration-c1       & 0.088                     & 0.001           & 0.002             & 0.013                          & 0.069                              \\ \hline
penetration-c2       & 0.086                     & 0.000           & 0.000             & 0.027                          & 0.001                              \\ \hline
password-crack-c1    & 0.186                     & 0.001           & 0.000             & 0.009                          & 0.000                              \\ \hline
password-crack-c2    & 0.183                     & 0.007           & 0.001             & 0.015                          & 0.001                              \\ \hline
password-crack-c3    & 0.256                     & 0.152           & 0.001             & 0.022                          & 0.000                              \\ \hline
data-leakage         & 0.568                     & 0.450           & 0.019             & 1.697                          & 0.031                              \\ \hline
command-injection-c1 & 0.246                     & 0.001           & 0.001             & 0.020                          & 0.000                              \\ \hline
command-injection-c2 & 0.215                     & 0.025           & 0.006             & 0.668                          & 0.008                              \\ \hline
vpnfilter-c1         & 0.179                     & 0.002           & 0.000             & 0.006                          & 0.000                              \\ \hline
vpnfilter-c2         & 0.162                     & 0.007           & 0.000             & 0.053                          & 0.000                              \\ \hline
\textbf{avg}         & \textbf{0.21}            & \textbf{0.06}  & \textbf{0.003}   & \textbf{0.25}                 & \textbf{0.01}                     \\ \hline
\end{tabular}
}
\end{table}

\begin{table*}[]
\centering
\caption{Execution time of \tool }
\resizebox{0.96\textwidth}{!}{
\begin{tabular}{c|r|r|r|r|r|r}
\hline
\multirow{2}{*}{\textbf{Case}} & \multicolumn{1}{c|}{\multirow{2}{*}{\textbf{Causality Ana.(s)}}} & \multicolumn{1}{c|}{\multirow{2}{*}{\textbf{Edge Merge(s)}}} & \multicolumn{2}{c|}{\textbf{Weight Computation(s)}}                  & \multicolumn{2}{c}{\textbf{Propagation(s)}}                        \\ \cline{4-7} 
                                   & \multicolumn{1}{c|}{}                                    & \multicolumn{1}{c|}{}                                        & \multicolumn{1}{c|}{\textbf{\tool}} & \multicolumn{1}{c|}{\textbf{Average Proj.}} & \multicolumn{1}{c|}{\textbf{\tool}} & \multicolumn{1}{c}{\textbf{Average Proj.}} \\ \hline
wget executable                    & 120.97                                                   & 0.05                                                         & 0.26                                 & 0.02                                 & 0.06                                 & 0.06                                \\ \hline
illegal storage                    & 92.86                                                    & 0.38                                                         & 7.43                                 & 0.39                                 & 19.48                                & 47.77                               \\ \hline
illegal storage 2                  & 95.13                                                    & 3.02                                                         & 52.68                                & 33.79                                & 160.08                               & 1,038.55                            \\ \hline
Hide File                          & 223.63                                                   & 42.16                                                        & 463.68                               & 16.14                                & 1,150.35                             & 8,486.32                            \\ \hline
Steal Information                  & 129.82                                                   & 39.51                                                        & 479.02                               & 15.98                                & 1,157.45                             & 8,128.28                            \\ \hline
Backdoor Download                  & 19.74                                                    & 0.44                                                         & 13.87                                & 0.32                                 & 12.75                                & 24.05                               \\ \hline
Annoying Server User               & 17.23                                                    & 0.01                                                         & 0.18                                 & 0.01                                 & 0.03                                 & 0.03                                \\ \hline
Shellshock                         & 0.05                                                     & 0.03                                                         & 0.07                                 & 0.01                                 & 0.02                                 & 0.06                                \\ \hline
Dataleak                           & 0.09                                                     & 0.01                                                         & 0.28                                 & 0.02                                 & 0.14                                 & 0.16                                \\ \hline
Vpnfilter                          & 0.28                                                     & 0.04                                                         & 0.35                                 & 0.03                                 & 0.11                                 & 0.14                                \\ \hline
\textbf{Avg.}                                & 69.98                                                    & 8.56                                                         & 101.78                               & 6.67                                 & 250.05                               & 1,772.54                            \\ \hline
\end{tabular}
}
\label{tab:rq4performance}
\end{table*}



\subsection{System Performance}
\eat{
To understand the performance of \tool in investigating real attacks, we measure the execution time of \tool on the attack cases.
\tool starts the computation by parsing a log ($92.252s$ averagely) and building a global graph representation ($3.277s$ averagely).
\cref{tab:runtime} shows the execution time for the remaining components of \tool. 
Besides the steps shown in the preprocessing step, \emph{Causality Analysis}, \emph{Edge Merge}, and \emph{Node split} require $0.21s$, $0.06s$, and $0.003$ on average. 
Note that \emph{Weight Computation} and \emph{Reputation Propagation} only requires $0.25s$ and $0.01s$ on average.
In summary, the total time for running an analysis is about $2$ minutes, but the major cost (\ie log parsing) can be improved by adopting caching or database indexing~\cite{gao2018aiql}.
}

To understand the performance of \tool in attack investigation, we measure the execution time of \tool. \cref{tab:rq4performance} shows the time cost of \tool. 
\tool starts the computation by building a dependency graph (\ie Causality Analysis). This step requires $69.98s$ on average. 
\tool takes averagely $8.56s$ to finish the graph prepossessing. 

We also compare \tool and the average-projection approach in the time cost needed for weight computation and relevance propagation. 
We can see that \tool needs more time to do the weight computation, because \tool use the clustering algorithm to find the optimal parameters for the projection-vector normalization. 
The shorter weight computation time for the average-projection approach is at the cost of a much longer time needed for the relevance propagation. 
By combining the time needed for weight computation and relevance propagation, \tool reduces the time cost by $80.23\%$ compared with the average-projection approach. 
On average, \tool requires $430s$ to finish the processing and computation for one attack.
