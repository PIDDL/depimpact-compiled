\begin{table}[!t]
\centering
\caption{Fanout weight distribution}
\resizebox{0.48\textwidth}{!}{
\begin{tabular}{l|l|l|l|l|l|l|l|l|l|l}
\hline
                            & \multicolumn{1}{c|}{\textbf{0}} & \multicolumn{1}{c|}{\textbf{0.1}} & \multicolumn{1}{c|}{\textbf{0.2}} & \multicolumn{1}{c|}{\textbf{0.3}} & \multicolumn{1}{c|}{\textbf{0.4}} & \multicolumn{1}{c|}{\textbf{0.5}} & \multicolumn{1}{c|}{\textbf{0.6}} & \multicolumn{1}{c|}{\textbf{0.7}} & \multicolumn{1}{c|}{\textbf{0.8}} & \multicolumn{1}{c}{\textbf{0.9}} \\ \hline
\multicolumn{1}{c|}{Fanout} & $\sim606k$             & $\sim5k$                 & $\sim4k$                 & $\sim10k$                & 0.00                     & $\sim35k$                & 0.00                     & 0.00                     & 0.00                     & $\sim45k$               \\ \hline
\end{tabular}
}
\label{tab:fanout}
\end{table}
\subsubsection{RQ4: Comparison with State-Of-The-Art Causality Analysis}
In this RQ, we compare \tool with the fanout approach used in the state-of-the-art causality analysis approach, PrioTracker~\cite{liu2018priotracker}. 
PrioTracker mainly uses the fanout of nodes to prioritize the dependencies in the causality analysis. 
We then adapt the computed priories as the edge weights and apply the graph reduction based on these weights. 
In this way, we can do a fair comparison between \tool and the fanout approach in graph reduction.
\cref{tab:fanout} shows the average number of edges belongs to each weight group for 10 attacks. 
Columns 0 represent the weight range from 0.0 to 0.1 (\ie $[0.0, 0.1\}$), similarly, the rest columns represent from 0.1 to 0.9. 
If we use the fanout weight to filter edges, the average number of edge having high weight (\ie $\ge 0.9$) is $45,409.84$. 
The number of edge is 215 times larger than the average number of edges processed by \tool, when it uses 3 entry nodes among the top 9 candidates to do the forward causality analysi for reduction. 
Thus, from the view of graph reduction, the fanout method is not effective as \tool.
