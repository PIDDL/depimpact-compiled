\section{Discussion}
\label{sec:discussion}

\eat{
\subsection{General Framework for Revealing More Attacks}
\tool provides a general framework to assign weights to dependency edges based on a set of features, 
and computes relevance scores based on edge weights.
Our evaluations have shown the effectiveness of \tool using the features (\cref{subsubsec:feature-extraction}) for the commonly used exploits and real attacks.
To reveal the types of attacks that present different properties, a different set of features that describe these properties can be provided to \tool,
and the discriminative feature projection will be able to automatically compute new weights for revealing the attacks. 
}

\subsection{Adversarial Settings}
As briefly discussed in \cref{subsubsec:feature-extraction}, in practice, the attacker, with some knowledge about the proposed system, may optimize its attack strategy to manipulate the proposed features of certain edges. 
For example, (1) to have a lower temporal relevance feature, the attacker may extend its malicious activities during weeks/months to remain stealth, or (2) to have a lower data size relevance feature, the attacker may perform the exfiltration using multiple processes and each process is only associated with small data size. 
%
However, note that the goal of this work is not to design highly robust features to accurately detect malicious activities, which is an orthogonal research question and is very challenging.
Instead, this work targets the design of an effective approach that identifies parts of dependencies that are actually related to a given POI, which is challenging due to the large imbalance between critical edges and non-critical edges.
%
As shown in \cref{subsubsec:rq1}, directly using the features to identify critical edges can lead to many false positives, which in turn demonstrates that it is fairly challenging to design highly effective features.
In contrast, \tool employs novel methods to take advantage of the noisy features (\ie edge weight computation, relevance score propagation, entry node ranking, forward causality analysis) and achieves a much higher graph reduction rate.
%


%\pgao{First, a given POI might correspond to an event that has no data amount attribute; like a rename system call or an unlink system call for deleting a critical file. }


%\pgao{fanout feature; priotracker => does not prune, but prioritize}



\subsection{Design Alternatives}
For feature extraction, besides the features proposed, the design of \tool supports easy incorporation of other features according to specific forensic investigation needs.
%
For edge weight computation, one alternative is to train a binary classifier using the features and output a probability score as the edge weight.
However, such supervised learning-based approach faces significant limitations in our problem context:
(1) as some of our features are computed with respect to the specific POI, the classification model learned for one type of attack can hardly generalize to other types of attacks with different POIs;
(2) such approach typically requires large amount of training data, while our problem context is highly imbalanced in which critical edges are limited. 
%
Among unsupervised learning-based approaches, approaches based on anomaly detection~\cite{anomalysurvey} could be a substitution for KMeans clustering, and there could be alternatives for LDA to achieve discriminative dimensionality reduction~\cite{Mika99fisherdiscriminant,sugiyama2006local}.
We plan to explore these options in future work.

%\pgao{maybe talk about local clustering approach}


\subsection{Parallel Execution}
Many parts of \tool can be potentially parallelized with distributed computing.
Backward and forward causality analyses can be parallelized by searching the dependency separately. 
Feature extraction for different edges is independent and can be easily parallelized.
%In the scenarios where multiple hosts are involved, dependencies on each host can be precomputed in parallel and thus cross-host causality analysis becomes the concatenation of multiple generated dependency graphs. 
Relevance score propagation (\cref{eq:reputation}) can be converted into a matrix-vector product form to save CPU cycles.
Further parallelization is possible by leveraging ideas similar to parallelizing PageRank~\cite{gleich2004fast,kohlschutter2006efficient}. 
We plan to explore these ideas in future work.



%Weighted causality analysis with parallelization is an interesting research direction that requires non-trivial efforts




\eat{
The construction of
causality graphs can be potentially parallelized with distributed
computing. Any individual branch to be explored can be
processed separately; branches may bear different priorities
and therefore are assigned with corresponding computing
resources; dependencies on each host can also be pre-computed
in parallel and cross-host tracking thus becomes the concatenation of multiple generated graphs. Nonetheless, the massive
and pervasive dependencies among system events bring significant challenges to parallel processing, and therefore distributed
causality tracking by itself is an interesting research direction
that requires non-trivial efforts.
}



\eat{
\subsection{Industrial View}
Because APT attacks consist of many small steps over a long period of time, even though security experts can obtain the system-wide log, it is time-consuming to manually inspect the daunting number of edges in the system dependency graph, and thus it is hard to discover the complete attack steps and reconstruct the attack sequence.
Moreover, depending on the individual's capability, the quality of the analysis may vary a lot. 
By enabling automatic investigation, \tool not only reduces the time consumption of the analysis but also mitigates the dependency on the capability of the security analysts.
This makes \tool highly applicable to small-scale businesses that are not affordable to hire a large team of security analysts to conduct labor-intensive investigation. 
%The automated process is vital to keep the security level as high as possible. 
% In places where there are multiple servers, such as a large-scale business, this work can guarantee a level of security through automation and help security experts to respond promptly. The prompt response will help to reduce the financial damage when a security incident happens.
}
