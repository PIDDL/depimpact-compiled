
%% \begin{table}[!hbt]
% \centering
% \caption{POI reputation under uniform library reputation}
% \begin{tabular}{|l|r|r|}
% \hline
% Case                    & Uni. Library & Uni.Lib \& Special. \\ \hline
% curl                    & 0.99     & 0.91     \\ \hline
% mal\_script             & 0.95     & 0.83     \\ \hline
% python\_wget            & 0.99     & 0.87     \\ \hline
% scp                     & 1.00     & 0.92     \\ \hline
% shell\_wget             & 0.98     & 0.88     \\ \hline
% 3File                   & 1.00     & 0.97     \\ \hline
% 2File                   & 1.00     & 0.88     \\ \hline
% wget                    & 1.00     & 0.99     \\ \hline
% cmd\_inject\_step3      & 1.00     & 0.82     \\ \hline
% cmd\_inject\_step5      & 0.97     & 0.90     \\ \hline
% shellshock\_leak        & 1.00     & 0.83     \\ \hline
% shellshock\_pass\_step1 & 1.00     & 0.85     \\ \hline
% shellshock\_pass\_step4 & 0.99     & 0.92     \\ \hline
% shellshock\_pass\_step5 & 0.99     & 0.89     \\ \hline
% shellshock\_pene\_step1 & 0.97     & 0.85     \\ \hline
% shellshock\_pene\_step3 & 0.95     & 0.79     \\ \hline
% vpn\_filter\_step1      & 0.99     & 0.80     \\ \hline
% vpn\_filter\_step5      & 1.00     & 0.86     \\ \hline
% \end{tabular}
% \label{tab:diffSeedReputation}
% \end{table}
\begin{table}[!t]
\centering
\caption{POI reputation of different reputation schemes}
\resizebox{0.45\textwidth}{!}{
\begin{tabular}{|l|r|r|r|}
\hline
\multirow{2}{*}{\textbf{Case}}              &    \multicolumn{1}{c|}{\textbf{Uniform Lib}}   & \multicolumn{2}{c|}{\textbf{Uniform All}} \\ \cline{2-4}
             & \multicolumn{1}{c|}{\textbf{POI}} & \multicolumn{1}{l|}{\textbf{POI}} & \multicolumn{1}{l|}{\textbf{Avg. Non-Lib}}  \\ \hline
curl                    & 0.99     & 0.91               & 0.91      \\ \hline
mal\_script             & 0.95     & 0.83               & 0.87      \\ \hline
python\_wget            & 0.99     & 0.87               & 0.88      \\ \hline
scp                     & 1.00     & 0.92               & 0.92      \\ \hline
shell\_wget             & 0.98     & 0.88               & 0.89      \\ \hline
3File                   & 1.00     & 0.97               & 0.97      \\ \hline
2File                   & 1.00     & 0.88               & 0.88      \\ \hline
wget                    & 1.00     & 0.99               & 0.99      \\ \hline
command-injection-c1*      & 1.00     & 0.82               & 0.82      \\ \hline
command-injection-c2*      & 0.97     & 0.90               & 0.93      \\ \hline
data-leakage*        & 1.00     & 0.83               & 0.83      \\ \hline
password-crack-c1* & 1.00     & 0.85               & 0.85      \\ \hline
password-crack-c2* & 0.99     & 0.92               & 0.93      \\ \hline
password-crack-c3* & 0.99     & 0.89               & 1.00      \\ \hline
penetration-c1* & 0.97     & 0.85               & 0.87      \\ \hline
penetration-c2* & 0.95     & 0.79               & 0.83      \\ \hline
vpnfilter-c1*      & 0.99     & 0.80               & 0.80      \\ \hline
vpnfilter-c2*      & 1.00     & 0.86               & 0.86      \\ \hline
\textbf{avg}      & \textbf{0.99} &\textbf{0.88} &\textbf{0.89} \\ \hline
\end{tabular}
}
\label{tab:diffSeedReputation}
\end{table}
\begin{table}[t]
\centering
\caption{Average rank of attack entries}
\label{tab:rq3}
\resizebox{0.5\textwidth}{!}{
\begin{tabular}{crrrrr}
\hline
 
\textbf{Attack}        & \multicolumn{1}{c}{\textbf{\tool-{}-}} & \multicolumn{1}{c}{\textbf{\tool-}} & \multicolumn{1}{c}{\textbf{Avg. Proj.}} & \multicolumn{1}{c}{\textbf{Rand.}} & \multicolumn{1}{c}{\textbf{\tool}} \\ \hline
Wget Executable            & 5.50                                     & 12.25                                          & 20.00                                    & 23.45                                    & 5.50                                 \\
Illegal Storage            & 25.00                                    & 13.00                                          & 18.00                                    & 475.99                                   & 7.00                                 \\
Illegal Storage2           & 1.00                                     & 1.00                                           & 1.00                                     & 1,893.66                                 & 1.00                                 \\
Hide File                  & 22.00                                    & 10.50                                          & 13.50                                    & 17,284.72                                & 1.00                                 \\
Steal Information          & 11.00                                    & 3.50                                           & 7.00                                     & 17,304.32                                & 11.00                                \\
Backdoor Download          & 3.50                                     & 3.50                                           & 7.50                                     & 76.57                                    & 6.50                                 \\
Annoying Server User\_user & 5.00                                     & 5.00                                           & 13.00                                    & 15.82                                    & 4.00                                 \\
Shellshcok                 & 11.00                                    & 19.00                                          & 13.00                                    & 22.63                                    & 11.00                                \\
Dataleak                   & 35.00                                    & 9.00                                           & 9.00                                     & 48.34                                    & 8.00                                 \\
VPN Filter                 & 46.00                                    & 34.00                                          & 8.00                                     & 236.77                                   & 7.00                                 \\
Five Dir. Case1            & 5.00                                     & 5.00                                           & 5.00                                     & 115.50                                   & 5                                    \\
Five Dir. Case3            & 1                                        & 1                                              & 1                                        & 327.10                                   & 2                                    \\
Theia Case1                & 1                                        & 1                                              & 1                                        & 88956.7                                  & 1                                    \\
Theia Case3                & 1                                        & 2                                              & 1                                        & 70610.5                                  & 1                                    \\
Trace Case5                & 2                                        & 2                                              & 2                                        & 10.1                                     & 2                                    \\
AVG                        & 11.67                                    & 8.12                                           & 8.00                                     & 13,160.14                                & 4.87                                 \\ \hline
\end{tabular}
}
\end{table}

\subsection{RQ3: Revealing Attack Entries}
\label{subsec:rq3}

The evaluation results in \cref{subsec:rq2} show that ranking attack entries at the top is critical for \tool to achieve good performance (very low $M_{miss}$ and low $M_{remain}$).
%
In this RQ, we aim to measure the effectiveness of \tool in revealing attack entries (\ie whether the attack entries are among the top-ranked entry nodes).
In particular, we compare \tool with 4 baseline approaches: the uniform random approach, 2 simplified versions of \tool: \tool-, \tool-{}-, and the average-projection approach.
The uniform random approach ranks all the entry nodes randomly.
\tool- uses the temporal relevance and the data size relevance to compute the dependency weight, but not the concentration ratio.
\tool-{}- uses only the temporal relevance to compute the  dependency weight.
The difference between \tool and the average-projection approach is the parameters used to
%normalize the features to 
form the projection vector. 
For the average-projection approach, we select a fixed parameter vector ($0.334, 0.333, 0.333$) to computer the dependency weight. %(\cref{eq:projection})


\cref{tab:rq3} shows the average ranks of all the attack entries computed by \tool and the baseline approaches.
We observe that \tool consistently ranks the attack entries at the top (average rank $3.13$) and achieves the best performance.
%
Compared with \tool-{}-, \tool-, the average-projection approach (shown in Column ``Avg. Proj.''), and the uniform random approach (shown in Column ``Rand.''), \tool achieves $48.69\%$, $46.95\%$, $29.34\%$ and $99.92\%$ improvement in ranking the attack entries.

% The high ranking performance of attack entries provides the foundation for \tool to generate a small subgraph (\ie critical component) that preserves all the critical edges. 
The comparison among \tool, \tool-{}-, and \tool- demonstrates the effectiveness of our three selected features,
and the comparison between \tool and the average-projection approach demonstrates the superiority of our discriminative feature projection scheme over the fixed average projection for computing the dependency weights. 



\eat{
versions, which only use feature temporal relevance or temporal relevance plus data size relevance. 
Column \emph{DEP-} shows the ranking result for only using temporal relevance. Column \emph{DEP+} shows the ranking result for using temporal relevance and data size relevance.
We also compare \tool with another weight-computation approach (average-projection) and the random selection. 
The difference between \tool and the average-projection approach is the parameter used to normalize the purposed features to the projection vector. 
For the average-projection approach, we manually select a set of fixed parameters (0.334, 0.333, 0.333) for the projection-vector normalization. 
Both \tool and the average-projection approach adopt the same score propagation method. 
After the relevance propagation, we rank entry nodes based on the relevance score. 
Ideally, the entry nodes relevant to the attack are expected to be ranked at the top. Based on the ground truth, we compute the average ranking of the attack-relevant entry nodes for these three approaches, as shown in~\cref{tab:rq3}. 

\label{subsubsec:reputationeval}
In previous evaluations, 
we have demonstrated the effectiveness of \tool when the reputations of entry nodes are well separated. 
Next, we would like to see how resilient \tool is when the reputation of entry nodes is mutated intentionally by attackers or unintentionally by mistake. 
Thus, we run the experiments on all the cases under two new reputation schemes:
(1) \textbf{Uniform Library}: assigning the reputation of a library node using a uniform random value in $\lbrack 0.2, 0.8\rbrack$, \ie making some of them malicious and some of them benign,
and keeping other nodes unchanged;
(2) \textbf{Uniform All}: besides shuffling the reputation uniformly for libraries, assigning the reputation of a trusted source using a uniform random value in $\lbrack 0.8, 1.0\rbrack$, and using $\lbrack 0.0, 0.2\rbrack$ for a untrusted source, respectively.

\cref{tab:diffSeedReputation} shows the results of the two schemes.
Note that to easily compute the average, for the real attack cases (marked with $*$), we convert the POI reputation $p$ to $1 - p$, so it will be shown as close to 1 when it is actually close to 0.
As we can see, in the Uniform Library scheme, for all cases, the POI reputation is close to $1.0$, indicating that even with the noise introduced in the libraries, the edge weights of \tool prevents the adverse impact on the POI reputation. 
In the Uniform All scheme, the POI reputations of some cases have obvious changes (\eg mal\_script, python\_wget, shell\_wget \textit{etc.}). 
Column \textit{Avg. Non-Lib} shows the average reputation of non-lib entry nodes representing trusted or suspicious sources. The Pearson correlation between the reputations of the POI and the non-lib entry nodes is $0.885$, indicating a strong correlation.

These results indicate that \tool is resilient to mutations on the library reputation and the POI reputation is strongly correlated with non-lib entry nodes.
In real world scenarios,
it will be relatively easy for the attacker to perturb the reputation for libraries given the massive number of libraries to keep track of. 
Also, since there are a few non-lib entry nodes in each attack and security analysts will be asked to inspect them, it will be more difficult for the attacker to mutate the reputation without being noticed.
If the attacker succeeds in tricking the security analysts in trusting a suspicious source, the security analyst will misclassify the attack since in the end the security analyst will determine whether it forms an attack based on the source,
no matter \tool is used or not.
}


