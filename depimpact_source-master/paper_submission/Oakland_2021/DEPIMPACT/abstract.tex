\begin{abstract}

Causality analysis on system auditing data has emerged as an important solution for attack investigation.
Given a POI (Point-Of-Interest) event (\eg an alert for creating a suspicious file), causality analysis constructs a dependency graph, where nodes represent system entities (\eg processes and files) and edges represent dependencies among entities, to reveal the attack sequence.
However, causality analysis often produces a huge graph ($> 100,000$ edges) that is hard for security analysts to inspect.
From the dependency graphs of various attacks, we observe that 
(1) dependencies that are highly related to the POI event often present a different set of properties (\eg data flow and time) compared to less-relevant dependencies;
(2) the POI event is often related to a few attack entries (\eg downloading a file).
Based on these insights, we propose \tool, a framework that identifies the critical component of a dependency graph (\ie a subgraph) by (1) assigning \textit{discriminative dependency weights} to edges to distinguish \textit{critical edges that represent the attack sequence} from less-important dependencies, (2) propagating dependency impacts backward from the POI event to entry points, and (3) ranking entry points by their dependency impacts.
In particular, \tool performs forward causality analysis from the top-ranked entry points that are likely to be the attack entries to filter out edges from the original dependency graph that are not found in the forward causality analysis.
Our evaluations on the 100 million real system auditing events of 10 attacks show that
\tool can significantly reduce the large dependency graphs ($\sim700,000$ edges) to a small graph ($\sim260$ edges), which is $\sim2,700\times$ smaller.
The comparison of \tool with other four state-of-the-art causality analysis techniques shows that \tool is at least $33\times$ more effective in reducing the dependency graphs for revealing the attack sequences.



\end{abstract}