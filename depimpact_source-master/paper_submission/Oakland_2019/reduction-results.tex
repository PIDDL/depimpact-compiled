\subsubsection{Graph Reduction Results}
\label{subsec:graphreduction}

\begin{table}[]\
\centering
    	\caption{Graph reduction results for PHASEs I and II}
    	\label{tab:Reduction}
    	\resizebox{0.48\textwidth}{!}{%
            \begin{tabular}{|l|r|r|r|r|}
            \hline
            \thead{Attacks}                & \thead{PH.I V} & \thead{PH.I E} & \thead{PH.II V} & \thead{PH.II E} \\ \hline
            3File      & 225                  & 25211             & 225            & 596         \\ \hline
            2File             & 221                  & 21100             & 221            & 537         \\ \hline
            curl                  & 212                  & 15142             & 212            & 460         \\ \hline
            shell\_script           & 229                  & 21124             & 229            & 637         \\ \hline
            python\_wget          & 162                  & 10597             & 162            & 283         \\ \hline
            scp                   & 60                   & 7912              & 60             & 78          \\ \hline
            shell\_wget           & 115                  & 4998              & 115            & 158         \\ \hline
            wget                  & 217                  & 19078             & 217            & 500         \\ \hline
            command-injection-c1      & 51                   & 65                & 51             & 51          \\ \hline
            command-injection-c2      & 1438                 & 9713              & 1438           & 2738        \\ \hline
            data-leakage       & 4252                 & 151200            & 4252           & 4261        \\ \hline
            password-crack-c1 & 35                   & 731               & 35             & 36          \\ \hline
            password-crack-c2 & 61                   & 2477              & 61             & 73          \\ \hline
            password-crack-c3 & 25                   & 58426             & 25             & 36          \\ \hline
            penetration-c1 & 60                   & 314               & 60             & 63          \\ \hline
            penetration-c2 & 30                   & 139               & 30             & 68          \\ \hline
            vpnfilter-c1      & 14                   & 274               & 14             & 14          \\ \hline
            vpnfilter-c2      & 16                   & 598               & 16             & 18          \\ \hline
            avg & 412.39 & 19394.39 & 412.39 & 589.28 \\ \hline
            \end{tabular}
        }
    % \dcaption{ .}
\end{table}


\cref{tab:Reduction} shows the reduction in the number of nodes and in the number of edges after causality analysis (\cref{subsec:graph-generation}) and edge merge (\cref{subsec:graph-preprocessing}).
As we can see, the reduction is significant: (1) In most of the cases, \tool achieves more than half of nodes reduced. Causality analysis helps trim up to 72.8\% nodes on average.
(2) In most of the cases, \tool achieves more than 95\% of edges reduced. Edge merge helps trim up to 97.74\% edges on average.

To surface critical edges, \tool uses a threshold to hide non-critical edges.
To provide a guidance on selecting this threshold, we test the filtering performance by selecting an increasing multiple of average weight of the whole graph from 0 to 2 with a pace of 0.05.
We define the \emph{threshold} as the average weight of the whole graph magnified by a number $T_w$ (\ie threshold multiplier). 
%
\cref{fig:edge-thresh} shows the average percentage of edges remaining of all cases after filtering. We observe a turning point at $T_w = 0.15$ and the number of remaining edges will remain stable below 20\%. Higher thresholds can lead to more graph size reduction. However, if we choose the threshold too high, we will lose track of some of the critical edges. 
%
We define the \emph{missing point} as the exact threshold multiplier that leads to the first critical edge loss(\cref{tab:filter}). 
\cref{fig:cdf} shows the cumulative distribution of missing points.
We observe that: 
(1) Two cases (\emph{command-injection-c2}, \emph{data-leakage}) have extremely high missing points ($T_w > 200$);
(2) 5 out of 21 cases lost critical edges at $T_w = 2$. However, in these 5 cases, 2 of them(\emph{Shell-wget},\emph{penetration-c1}) already have less than 10 non-critical edges at missing point and 3 of them also have significant reduction in edge numbers(\cref{tab:filter}).
(3) A plateau exists before $T_w = 2$ at a rate of 24\%. This indicate most of the cases have a missing point greater than $T_w = 2$, which proves the efficacy of our weights to differ critical edges from non-critical edges.

Given that setting $T_w = 0.15$ is enough to filter out more than 80\% of the non-critical edges and 76\% of the cases have $T_w > 2$. A good strategy would be examining the graph at $T_w = 2$ to grab a rough sense then tuning down to $T_w = 0.15$ to review details. To avoid critical edge miss in some situation, then going down to $T_w = 0$. Rather than directly examine the graph after Edge Merge, this will save a lot of daunting labor.
  


\begin{table}[]
\centering
        \caption{Filtering Results}
        \label{tab:filter}
        \resizebox{0.45\textwidth}{!}{%
            \begin{tabular}{|l|r|r|r|}
            \hline
            \thead{Cases} & \thead{\#Critical Edges} & \thead{Missing Point} & \thead{\#Non-critical Edges at Missing Point}\\\hline
            2File                     & 3                        & 9.49          & 0                                     \\\hline
            3File                     & 4                        & 6.49          & 0                                     \\\hline
            Python-wget               & 4                        & 5.19          & 1                                     \\\hline
            Python-wget-unzip         & 8                        & $<0.01$          & 60                                    \\\hline
            Shell-script              & 4                        & 3.15          & 1                                     \\\hline
            Shell-wget                & 4                        & 0.04          & 6                                     \\\hline
            Shell-wget-unzip          & 6                        & 2.83          & 3                                     \\\hline
            USB-merge                 & 6                        & 2.11          & 0                                     \\\hline
            curl                      & 4                        & 12.80         & 1                                     \\\hline
            scp                       & 3                        & 8.29          & 4                                     \\\hline
            wget                      & 2                        & 6.20          & 29                                    \\\hline
            command-injection-c1 & 2                        & 17.00         & 49                                    \\\hline
            command-injection-c2 & 3                        & 286.50        & 0                                     \\\hline
            data-leakage             & 5                        & 302.89        & 0                                     \\\hline
            password-crack-c1    & 2                        & 9.00          & 34                                    \\\hline
            password-crack-c2    & 4                        & 14.20         & 1                                     \\\hline
            password-crack-c3    & 4                        & $<0.01$          & 57                                    \\\hline
            penetration-c1         & 3                        & 0.02          & 5                                     \\\hline
            penetration-c2         & 11                       & $<0.01$          & 21                                    \\\hline
            vpnfilter-c1          & 2                        & 4.67          & 12                                    \\\hline
            vpnfilter-c2          & 3                        & 3.60          & 15                                   \\\hline 
            \thead{average}    & 4                        & 32.92          &14.24 \\\hline
            \end{tabular}
        }
\end{table}
\begin{figure}[hbt!]
    \centering
    \includegraphics[width=0.45\textwidth]{figs/fig:edge-thresh.png}
    \caption{Effectiveness of Filtering}
    \dcaption{The percentage of edges remaining after filtering drops significantly at $T_w = 0.15$ and remains stable below 20\% (\ie filtering threshold equals $T_w$ multiplies the average weight of all edges).}
    \label{fig:edge-thresh}
\end{figure}
\begin{figure}[hbt!]
    \centering
    \includegraphics[width=0.48\textwidth]{figs/fig:cdf.png}
    \caption{Critical Edge Loss from Filtering}
    \dcaption{Missing points distribute mostly between $T_w = 2$ and $T_w = 18$. Note a plateau before $T_w = 2$.}
    \label{fig:cdf}
\end{figure}
